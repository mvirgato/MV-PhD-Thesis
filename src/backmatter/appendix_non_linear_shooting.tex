%%%%%%%%%%%%%%%%%%%%%%%%%%%%%%%%%%%%%
%%%%%%%%%%%%%%%%%%%%%%%%%%%%%%%%%%%%%
\chapter{Solving the FMT EoS}
\label{app:FMT_solution}
%%%%%%%%%%%%%%%%%%%%%%%%%%%%%%%%%%%%%
%%%%%%%%%%%%%%%%%%%%%%%%%%%%%%%%%%%%%

Here we demonstrate how the nonlinear shooting method is used to solve the zero-temperature Feynman-Mettopolis-Teller equation of state for white dwarfs. The same algorithm can be used to solve the finite temperature EoS. 

To begin, define a function $f(x, \chi, \chi')$ to be
\begin{align}
    f(x, \chi, \chi') & = \frac{d^2\chi}{dx^2}\\
    & =  -\frac{3 x \alpha_\mathrm{EM}}{\Delta^3}\theta(x_c -x) + \frac{4 x\alpha_\mathrm{EM}}{3\pi}\left[ \frac{\chi^2(x)}{x^2} + 2\frac{m_e}{m_\pi}\frac{\chi(x)}{x}\right]^{3/2},
 \end{align}   
with the domain $0 \leq x \leq x_\mathrm{WS}$.
In general, the boundary conditions can be written as a set of linear equations
\begin{align}
    a_0 \chi(0) - a_1 \chi'(x_\mathrm{WS}) &= \gamma. \\ 
    b_0 \chi(0) + b_1 \chi'(x_\mathrm{WS}) & = \beta
\end{align}
where in the present case the coefficients are simply
\begin{equation}
    a_0 = 1,\quad a_1 = 0,\quad b_0 = 1,\quad b_1 = -x_\mathrm{WS}, \quad \gamma =0,\quad \beta = 0.
\end{equation}
We will remain as general as possible so that this method can be adapted easily to other systems while highlighting the key results for the current application.

This boundary value problem (BVP) can be related to an equivalent initial value problem (IVP) by parameterising the initial slope by the paramater $s$.  We define for the function $u(x)$ to be a solution to this IVP such that
\begin{equation}
    u''(x) = f(x, u, u'),
\end{equation}
with the initial conditions
\begin{align}
    u(0) & = a_1 s - c_1 \gamma \\
    u'(0) & = a_0 s - c_0 \gamma 
\end{align}
where the coefficients $c_0$ and $c_0$ are any constants that satisfy 
\begin{equation}
    a_1 c_0 - a_0 c_1 = 1.
\end{equation}
For the FMT case, we have $c_1 = -1$, and $c_0$ is a free parameter that can be set to 0, leaving
\begin{align}
     u(0) & = 0\\
     u'(0) & = s - c_0.
\end{align}

The solution to the IVP depends on the value of $s$, and so we write this as $u(x;s)$. This solution will also satisfy the original BVP if the $s$ is a root of
\begin{align}
    &b_0 u(b;s) + b_1 u'(b; s) - \beta = 0\\
     \implies &  u(x_\mathrm{WS};s) - x_\mathrm{WS} u'(x_\mathrm{WS};s) = 0\quad\mathrm{(FMT\; EoS)}.
\end{align}
It will be convenient to express this condition in terms of the roots of a function $\phi(s)$ defined as 
\begin{equation}
    \phi(s) = b_0 u(b;s) + b_1 u'(b; s) - \beta = 0.
\end{equation} 

The IVP can be solved by reduction of order, changing it into a system of two first-order non-linear ODEs taken to be 
\begin{align}
    u(x)' & = v(x),\\
    v'(x) & = f(x, u, v),
\end{align}
with the initial conditions given in general by
\begin{align}
    u(a) & = a_1 s - c_0 \gamma, \\
    v(a) & = a_0 s - c_0 \gamma, 
\end{align}
and specifically here are
\begin{align}
    u(0) &= 1,\\
    v(0) & = s - c_0.
\end{align}
We denote the solutions as $u(x;s)$ and $v(x;s)$, and re-express $\phi(s)$ in terms of these functions as
\begin{align}
     \phi(s) & = b_0 u(b;s) + b_1 v(b; s) - \beta = 0.
\end{align}


To solve the system numerically, we can define a grid consisting of $J$ points 
\begin{equation}
    0\leq x_j \leq x_\mathrm{WS},\quad 0\leq j\leq J,
\end{equation}
and solve the system on this grid for some initial choice of $s_0$. We represent the numerical solutions as vectors $U_j(s_0)$,  $V_j(s_0)$, with $\phi(s)$ is then simply
\begin{align}
    \phi(s_0) & = b_0 U_J(s_0) +b_1 V_J(s_0) - \beta.\\
\end{align}
As this choice of $s_0$ was essentially arbitrary, it will almost certainly not be a root. We can then use Newton's method to iteratively select the next $s$ used to solve the system. In general, the next iteration of $s$ given the previous choice is found through
\begin{align}
    s_{i+1} & = s_i - \frac{\phi(s_i)}{\dot{\phi}(s_i)}
\end{align}
where $\dot{\phi}(s) = \frac{\partial\phi}{\partial s}$.

However, $\dot{\phi}(s)$ is trivial to calculate, as it depends on the functions $u(x;s)$ and $v(x;s)$. We must therefore define an additional set of functions, 
\begin{align}
    \xi(x) & = \frac{\partial u(x;s)}{\partial s},\\
    \eta(x) & = \frac{\partial v(x;s)}{\partial s},
\end{align}
that results in the additional ODEs that must be solved alongside the original two,
\begin{align}
    \xi' & = \eta,\\
    \eta' & = p(x;s)\eta +q(x;s) \xi,
\end{align}
with the $p$ and $q$ functions given by
\begin{align}
    p(x;s) & = \frac{\partial f}{\partial v},\\
    q(x;s) & = \frac{\partial f}{\partial u}.
\end{align}

The initial conditions for these equations are 
\begin{align}
    \xi(a) & = a_1, \\
    \eta(a) & = a_0,
\end{align}

For the application to the FMT EoS, these functions are given explicitly by
\begin{align}
    p(x;s) & = 0.\\
    q(x;s) & = \frac{4\alpha_\mathrm{EM} }{\pi}\left( \frac{u^2}{x^2} + 2\frac{m_e}{m_\pi}\frac{u}{x} \right)^{1/2}\left( \frac{u}{x} + \frac{m_e}{m_\pi} \right),
\end{align}
Thus we can now express $\dot \phi (x)$ in terms of these new functions as
\begin{align}
    \dot{\phi}(s) &= b_0 \xi(b;s) + b_1\eta(b;s).\\
\end{align}
The complete system of ODEs needed to solve the IVP is given by  
\begin{align}
    u' & = v,\\
    v' & = -\frac{3 x \alpha_\mathrm{EM}}{\Delta^3}\theta(x_c -x) + \frac{4 x\alpha_\mathrm{EM}}{3\pi}\left( \frac{u^2}{x^2} + 2\frac{m_e}{m_\pi}\frac{u}{x}\right)^{3/2},\\
    \xi' & = \eta\\
    \eta' & = \frac{4\alpha_\mathrm{EM} }{\pi}\left( \frac{u^2}{x^2} + 2\frac{m_e}{m_\pi}\frac{u}{x} \right)^{1/2}\left( \frac{u}{x} + \frac{m_e}{m_\pi} \right) \xi
\end{align}
with the conditions 
\begin{align}
    u(0) &=0,\quad v(0) =s_i,\\
    \xi(0) &=0,\quad \eta(0) =1,
\end{align}
where $s_i$ is the $i$th iteration of $s$ found by via
\begin{align}
    s_{i+1} = s_i - \frac{u(x_\mathrm{WS};s_i) - x_\mathrm{WS} v(x_\mathrm{WS};s_i)}{\xi(x_\mathrm{WS};s_i) - x_\mathrm{WS} \eta(x_\mathrm{WS};s_i)}.
\end{align}
The initial ``guess'' of $s_0$ that starts the loop should be chosen wisely to be somewhat close to the true value. The final system is found once the absolute value of $\phi(s_i)$ falls below the tolerance specified, 
\begin{equation}
    |\phi(s_i)|<\varepsilon_\mathrm{tol},
\end{equation}
with standard values of $\varepsilon_\mathrm{tol} = 10^{-4}$.

Once the system is solved, we re-associate $u(x)$ with $\chi(x)$ and the pressure, density and chemical potential profiles within the cell can be obtained.

Depending on the algorithm used in solving the IVP, the Jacobian matrix may be required and is given by
\begin{equation}
    J = \begin{pmatrix}
         0 & 1 & 0 & 0 \\ \frac{4\alpha_\mathrm{EM} }{\pi}\left( \frac{u^2}{x^2} + 2\frac{m_e}{m_\pi}\frac{u}{x} \right)^{1/2}\left( \frac{u}{x} + \frac{m_e}{m_\pi} \right) & 0 & 0 & 0 \\ 0 & 0 & 0 & 1 \\ \frac{8\alpha_\mathrm{EM} \xi }{\pi x}\frac{\left( \frac{u^2}{x^2} + 2\frac{m_e u}{m_\pi x} +\frac{m_e^2}{2m_\pi^2} \right)}{\left( \frac{u^2}{x^2} + 2\frac{m_e}{m_\pi}\frac{u}{x} \right)^{1/2} } & 0 & \frac{4\alpha_\mathrm{EM} }{\pi}\left( \frac{u^2}{x^2} + 2\frac{m_e}{m_\pi}\frac{u}{x} \right)^{1/2}\left( \frac{u}{x} + \frac{m_e}{m_\pi} \right) & 0
    \end{pmatrix},
\end{equation}
as well as the vector of the second derivatives
\begin{equation}
    \frac{\partial}{\partial x}
    \begin{pmatrix}
        u' \\ v' \\ \xi' \\ \eta'
    \end{pmatrix}
    =
    \begin{pmatrix}
         0 \\ -\frac{3\alpha_\mathrm{EM} }{\Delta^3}\theta(x_c - x) - \frac{8 \alpha_\mathrm{EM} }{3 \pi}\left( \frac{u^2}{x^2} + 2\frac{m_e}{m_\pi}\frac{u}{x} \right)^{1/2} \left( \frac{u^2}{x^2} + \frac{m_e}{2 m_\pi}\frac{u}{x} \right) \\ 0 \\ -\frac{8\alpha_\mathrm{EM} \xi }{\pi x^2}\frac{\left( \frac{u^2}{x^2} + 2\frac{m_e u}{m_\pi x} +\frac{m_e^2}{2 m_\pi^2} \right)}{\left( \frac{u^2}{x^2} + 2\frac{m_e}{m_\pi}\frac{u}{x} \right)^{1/2} })
    \end{pmatrix}.
\end{equation}