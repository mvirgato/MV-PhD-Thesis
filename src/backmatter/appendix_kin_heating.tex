%%%%%%%%%%%%%%%%%%%%%%%%%%%%%%%%%%%%%%%%%%%%%%
%%%%%%%%%%%%%%%%%%%%%%%%%%%%%%%%%%%%%%%%%%%%%%
\chapter{Kinetic Heating}
\label{appendix:kin_heating}
%%%%%%%%%%%%%%%%%%%%%%%%%%%%%%%%%%%%%%%%%%%%%%
%%%%%%%%%%%%%%%%%%%%%%%%%%%%%%%%%%%%%%%%%%%%%%

%%%%%%%%%%%%%%%%%%%%%%%%%%%%%%%%%%%%%%%%%%%%%%
\section{DM Orbits in General Isometric Metric}
%%%%%%%%%%%%%%%%%%%%%%%%%%%%%%%%%%%%%%%%%%%%%%

The metric at any point inside or outside the NS can be written as 
\begin{equation}
    ds^2 = B(r) dt^2 - A(r) dr^2 - r^2( d\phi + \sin\theta d\theta^2)
\end{equation}
Along an orbit, the conserved conjugate momenta are the angular momentum per unit mass, $p_\phi = -L$ 
and the energy per unit mass $p_t = E_\chi$, and taking the orbit to lie in the $\theta = \pi/2$ plane 
leads to $p_\theta = 0$. 

The equation which describes the orbit can be obtained from the square of the energy-momentum 4-vector,
\begin{align}
    g_{\alpha\beta} p^\alpha p^\beta - m_\chi^2 & = 0\\
    \implies g^{\alpha\beta} p_\alpha p_\beta - m_\chi^2 & = 0
\end{align}
with
\begin{equation}    
g^{tt} = 1/B(r),\quad g^{rr} = -1/A(r),\quad g^{\phi \phi} = -1/r^2
\end{equation}
\begin{align}
    \implies 0 & = g^{tt} p_t p_t + g^{rr} p_r p_r + g^{\phi\phi} p_\phi p_\phi - m_\chi^2 \\
    & = \frac{E_\chi^2}{B(r)} - \frac{1}{A(r)} \left( g_{rr'} p^{r'} \right)\left( g_{rr'} p^{r'} \right) - \frac{L^2}{r^2} - m_\chi^2 \\
    & = \frac{E_\chi^2}{B(r)} - m_\chi^2 A(r) \left( \frac{dr}{d\tau} \right)^2 - \frac{L^2}{r^2} - m_\chi^2
\end{align}

To find $dt/d\tau$, we use
\begin{align}
    p^t & = m_\chi \frac{dt}{d\tau} = g^{tt}p_t = \frac{E_\chi}{B(r)}\\
    \implies \frac{dt}{d\tau} & = \frac{1}{B(r)}\frac{E_\chi}{m_\chi}
\end{align}
% \Mcomm{In the papers we have $d\tau = \sqrt{B(r)} dt$, so where is the inconsistency?}

This gives
\begin{equation}
    \left (\frac{dr}{dt} \right)^2 = \frac{B}{\tilde E_\chi^2 A} \left[\tilde E_\chi^2- B(r) \left(  1 + \frac{\tilde L^2}{r^2} \right) \right]\label{eq:drdt2GR}
\end{equation}

% \Mcomm{with alternative $dt/d\tau$
% \begin{equation}
%     \left (\frac{dr}{dt} \right)^2 = \frac{1}{A} \left[ \tilde E_\chi - B(r) \left( 1 + \frac{\tilde L^2}{r^2} \right) \right]
% \end{equation}
% }

For simplicity, consider orbits that are a straight line ($\tilde L = 0$), which has a radial extent $R$. This is related to $\tilde E_\chi$ through
\begin{gather}
    \tilde E_\chi^2 = B(R)\label{eq:maxradgeneral}\\
    \implies R = \frac{2 G M_\star}{1 - \tilde E_\chi^2}, \quad R>R_\star
    \label{eq:MaxRadius}
\end{gather}
using $B(r>R_\star) = 1 - 2 G M_\star /r$.

It is important to note that $E_\chi$ so far has been the \textit{conserved} energy along the orbit, 
which for the initial approach is $E_\chi = m_\chi + \frac{1}{2}m_\chi u^2\sim m_\chi$. 
We now call this energy $E_\chi^{\rm orbit}$, which is related to the DM energy as seen by a distant observer, $E_\chi^{\rm int}$, 
and is the energy used in calculating the interaction rates, through 
\begin{equation}
    E_\chi^{\rm orbit} = \sqrt{g_{tt}} E_\chi^{\rm int} = \sqrt{B(r)}E_\chi^{\rm int}
\end{equation}
and as $E_\chi^{\rm orbit} < m_\chi$ for all subsequent scatters after capture, eq.~\ref{eq:MaxRadius} is always positive.

These ``orbits" are straight lines that pass through the star's centre and extend an amount $R - R_\star$ on either side. 
Due to the symmetry of the motion, the period of the orbit is then
\begin{equation}
    T_{\rm orbit} = 4 \int_0^R \frac{1}{dr/dt}dr
\end{equation}
More relevant to this application is the time spent inside and outside the star, which is given by
\begin{align}
    T_{\rm inside} & = 4 \int_0^{R_\star} \frac{1}{dr/dt}dr\label{eq:timeinside}\\
    T_{\rm inside} & = 4 \int_{R_\star}^R \frac{1}{dr/dt}dr\label{eq:timeoutside}
\end{align}

% %%%%%%%%%%%%%%%%%%%%%%%%%%%%%%%%%%%%%%%%%%%%%%

% \section{Keeping Angular Dependence}
% %%%%%%%%%%%%%%%%%%%%%%%%%%%%%%%%%%%%%%%%%%%%%%

% For $\tilde L \neq 0$, we need the equation of motion for the angular coordinate $\phi$,

% \begin{align}
%     \frac{d \phi}{d\tau } & = g^{\phi\phi}p_\phi = \frac{\tilde L}{r^2}\\
%     \implies \frac{d\phi}{dt} & = \frac{B(r)}{\tilde E_\chi}\frac{\tilde L^2}{r^2}
% \end{align}
% The maximum value of  $\tilde L^2$ is  given by
% \begin{equation}
%     \tilde L^2_{\rm MAX} = \frac{ \tilde E_\chi^2 - B(r)}{B(r)}r^2
% \end{equation}
% and we parameterise the possible angular momentum along the orbit as 
% \begin{equation}
%     \tilde L^2 = y \tilde L^2_{\rm MAX} ,\quad 0<y<1.
% \end{equation}
% leading to the equation for $dr/dt$ simplifying to 
% \begin{equation}
%     \frac{dr}{dt} = \frac{B(r)}{\tilde E_\chi^2 A(r)}\left( \tilde E_\chi^2 - B(r) \right)(1 - y)
% \end{equation}
% showing that the maximum radius of the orbit does not change from the $\tilde L = 0$ case, only the time spent outside the star changes.

% % \begin{align}
% %     \frac{dt}{d\phi} & = \left( \frac{dt}{dr}\right)
% % \end{align}

%%%%%%%%%%%%%%%%%%%%%%%%%%%%%%%%%%%%%%%%%%%%%%
\section{Checking Newtonian/Non-Relativistic Limit}
%%%%%%%%%%%%%%%%%%%%%%%%%%%%%%%%%%%%%%%%%%%%%%

In the Newtonian limit, we take 
\begin{align}
B - 1\approx2 \phi \ll 1,\\
A - 1 \approx - 2 G M(r) / r \equiv -2V(r)\ll 1,\\
\tilde L^2 /r^2 \ll 1,\\
\tilde E - 1 = \varepsilon \ll 1,
\end{align}
with $\varepsilon$ the non-relativistic energy per unit mass. Then expanding Eq.~\ref{eq:drdt2GR} we get
\begin{align}
    \left(\frac{dr}{dt}\right)^2 & = (1 + 2 \phi)(1 + 2V) - (1 + 2\phi)^2(1 + 2 V)(1 - 2 \varepsilon)\left(1 + \frac{\tilde L^2}{r^2}\right)\\
    & = 1 + 2 \phi + 2 V - \left(1 + 4 \phi + 2 V + \frac{\tilde L^2}{r^2} - 2 \varepsilon \right)\\
    & = -2 \phi - \frac{\tilde L^2}{r^2} + 2 \varepsilon\\
    \implies \frac{1}{2}\left(\frac{dr}{dt}\right)^2 & +\frac{\tilde L^2}{ 2 r^2} + \phi = \varepsilon
\end{align}
which is the standard result for a Newtonian orbit.

\section{Procedure for calculating kinetic heating time}

\begin{itemize}
    \item Select a point in the star for the DM to scatter off, $r_{\rm scatter, 0}$. 
    \item DM comes in from infinity with initial energy $E_\chi \approx m_\chi$
    \item Boost DM to local energy of $m_\chi/\sqrt{B(r_{\rm scatter})}$
    \item Scatter the DM and calculate initial $\Delta E_\chi$
    \item Set local DM energy to $E_\chi \equiv p^t = m_\chi/\sqrt{B(r_{\rm scatter})} - \Delta E_\chi$
    \item Calculate the new conserved energy per unit mass along the orbit as 
    \begin{equation}
        \tilde E_\chi^{\rm orbit} = \sqrt{B(r_{\rm scatter})}E_\chi/m_\chi = \frac{\sqrt{B(r_{\rm scatter})}}{ m_\chi} (m_\chi/\sqrt{B(r_{\rm scatter, 0})} - \Delta E_\chi)
    \end{equation}
    \item Use Equation~\ref{eq:maxradgeneral} to solve for the maximum radius of the orbit, $R_{\rm orbit}$. 
    \item Use equations~\ref{eq:timeinside} and~\ref{eq:timeoutside} to calculate $T_{\rm in} / (T_{\rm in} + T_{\rm out})$
    \item Adjust the time interval between scatter by $dt\rightarrow dt (T_{\rm in} / (T_{\rm in} + T_{\rm out}))^{-1}$
    \item Iterate until $R_{\rm orbit} < R_\star$
\end{itemize}