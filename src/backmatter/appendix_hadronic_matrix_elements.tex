\chapter{Hadronic matrix elements for scattering operators}
\label{app:hadronic_matrix_elements}
%%%%%%%%%%%%%%%%%%%%%%%%%%%%%%%%%%%%%%%%%%%%%%%%%%%%%%%%%%%%%%%%%%%%%%%%%%%%%%%%%%%%%%%%%%%%%%%%%%%%
%%%%%%%%%%%%%%%%%%%%%%%%%%%%%%%%%%%%%%%%%%%%%%%%%%%%%%%%%%%%%%%%%%%%%%%%%%%%%%%%%%%%%%%%%%%%%%%%%%%%
%%%%%%%%%%%%%%%%%%%%%%%%%%%%%%%%%%%%%%%%%%%%%%%%%%%%%%%%%%%%%%%%%%%%%%%%%%%%%%%%%%%%%%%%%%%%%%%%%%%%

This appendix provides the hadronic matrix elements required for computing the dark matter-baryon couplings given in Eqs.~\ref{eq:cBS}-\ref{eq:cBT}, for both nucleons and hyperons.


\section{Nucleons}
The values of the hadronic matrix elements for neutrons and protons used in this paper are listed in Table~\ref{tab:opers_defn_full}. The values of $\Delta_q^{(p)}$ (and similarly for $\delta_q^{(p)}$) are obtained using isospin symmetry: 
\begin{equation}
\Delta_u^{N} =\Delta_d^{N^\star},\qquad \qquad
\Delta_s^{N} = \Delta_s^{N^\star},
\end{equation}
where $N^\star$ is the nucleon obtained interchanging $u\Longleftrightarrow d$ quarks.

\begin{table}[th]
    \centering
    \begin{tabular}{|c|c|c|c|c|}
    \hline
     $q$ & $f_{T_q}^{(n)}$ \cite{Belanger:2013oya_MicrOMEGAs3programcalculating} & $f_{T_q}^{(p)}$ \cite{Belanger:2013oya_MicrOMEGAs3programcalculating} & $\Delta^{(n)}_q$ & $\delta^{(n)}_q$ \cite{Belanger:2013oya_MicrOMEGAs3programcalculating} \\ 
     \hline
     $u$  & 0.0110 & 0.0153     & -0.319 \cite{QCDSF:2011aa_Strangenesscontributionproton} & -0.230 \\ \hline
     $d$ & 0.0273  & 0.0191     & 0.787 \cite{QCDSF:2011aa_Strangenesscontributionproton} & 0.840 \\ \hline
     $s$ & 0.0447  & 0.0447     & -0.040 \cite{Dienes:2013xya_Overcomingvelocitysuppression} & -0.046 \\ \hline
    \end{tabular}
    \caption{Hadronic matrix elements for neutrons and protons. }
    \label{tab:hadmatelem}
\end{table}

\section{Hyperons}\label{apx:hypff}

To calculate the $f_{T_q}^{(\cal B)}$ couplings for hyperons, we use the baryonic  sigma terms from ref.~\cite{Shanahan:2013apa_Octetspinfractions}, listed in Table~\ref{tab:DeltaBar}, in the following way 
\begin{eqnarray}
f_{T_{u,d}}^{(\cal B)} &=& \frac{\sigma_{l \cal B}}{m_{\cal B}}\frac{m_{u,d}}{m_u+m_d},\\
f_{T_s}^{(\cal B)} &=& \frac{\sigma_{s}}{m_{\cal B}},
\end{eqnarray}
where the first relation assumes
\begin{equation}
    \frac{\sigma_u^{\cal B}}{m_u} = \frac{\sigma_d^{\cal B}}{m_d}.
\end{equation}
In addition, we assume
\begin{equation}
 \sigma_u^{\cal B} = \sigma_d^{{\cal B}^*}, \qquad \qquad
 \sigma_s^{\cal B} = \sigma_s^{{\cal B}^*}.
\end{equation}
%
For the dimension 6 operators where $c_q\propto m_q$, the nucleon couplings depend only on the following sum of the $f_{T_q}^{(\cal B)}$ values
\begin{eqnarray}
\sum_{q=u,d,s} f_{T_q}^{(\cal B)} = \frac{\sigma_{l \cal B}+\sigma_s}{m_{\cal B}},
\end{eqnarray}
and hence exact values for the individual $f_{T_q}^{(\cal B)}$ are unnecessary.
%
The axial vector \cite{Shanahan:2013apa_Octetspinfractions,Alexandrou:2020sml_Completeflavordecomposition} and tensor \cite{Zanotti:2017bte_Transversespindensities} couplings are listed in Table~\ref{tab:DeltaBar}. 

\begin{table}[htbp]
    \centering
    \begin{tabular}{|c|c|c|c|c|c|c|c|c|}
        \hline
         $\cal B$ & $\sigma_{l \cal B}$ (MeV) & $\sigma_s$ (MeV) & $\Delta_u$ & $\Delta_d$ & $\Delta_s$ & $\delta_u$ & $\delta_d$ & $\delta_s$ \\ \hline
         $\Lambda^0$ & $32\pm4$ & $176\pm19$ & 0 & 0 & 0.59 & 0 & 0 & 0.47 \\ \hline
         $\Xi^0$ & $13\pm 10$ & $334\pm21$ & -0.38 & 0 & 1.03 & -0.22 & 0 & 0.9  \\ \hline
         $\Xi^-$ & $13\pm 10$ & $334\pm21$ & 0 & -0.38 & 1.03 & 0 & -0.22 & 0.9 \\ \hline
    \end{tabular}
    \caption{Sigma commutators for scalar interactions for each hyperon $\cal B$ (first and second column). Spin matrix elements for axial vector~\cite{Shanahan:2013apa_Octetspinfractions,Alexandrou:2020sml_Completeflavordecomposition} and tensor interactions are given in the remaining columns. Tensor couplings for $\Xi$ are taken from ref.~\cite{Zanotti:2017bte_Transversespindensities}, while those for $\Lambda^0$ are our estimates.}
    \label{tab:DeltaBar}
\end{table}