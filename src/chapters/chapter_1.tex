\graphicspath{{img/chapter_1/}}

\chapter{Introduction}
\label{chapter:introduction}

\begin{synopsis}
Background on DM and its current status
\end{synopsis}
%%%%%%%%%%%%%%%%%%%%%%%%%%%%%%%%%%%%%
%%%%%%%%%%%%%%%%%%%%%%%%%%%%%%%%%%%%%
%%%%%%%%%%%%%%%%%%%%%%%%%%%%%%%%%%%%%

Dark Matter is an enigma in modern physics. Ever since it was first 
proposed by Fritz Zwicky nearly 90 years ago, significant scientific
effort has gone into trying to discern its nature. However, despite the
best efforts of generations of physicists, a definitive detection
proving its existence eludes us. Nevertheless, dark matter's influence 
on our universe is undeniable, with a large assortment of evidence 
supporting its existence. 


%%%%%%%%%%%%%%%%%%%%%%%%%%%%%%%%%%%%%
%%%%%%%%%%%%%%%%%%%%%%%%%%%%%%%%%%%%%
%%%%%%%%%%%%%%%%%%%%%%%%%%%%%%%%%%%%%
\section{Evidence for Dark Matter}
%%%%%%%%%%%%%%%%%%%%%%%%%%%%%%%%%%%%%
%%%%%%%%%%%%%%%%%%%%%%%%%%%%%%%%%%%%%
%%%%%%%%%%%%%%%%%%%%%%%%%%%%%%%%%%%%%

Today, the amount of evidence in support of dark matter's existence is overwhelming. 

The first sign that there was an additional component to the mass-energy component of the universe came from observations of galaxy rotation curves. It was noted by Zwicky~\cite{Zwicky:1937zza_MassesNebulaeClusters} that the 

%%%%%%%%%%%%%%%%%%%%%%%%%%%%%%%%%%%%%
%%%%%%%%%%%%%%%%%%%%%%%%%%%%%%%%%%%%%
%%%%%%%%%%%%%%%%%%%%%%%%%%%%%%%%%%%%%
\section{Potential Models of Dark Matter}
%%%%%%%%%%%%%%%%%%%%%%%%%%%%%%%%%%%%%
%%%%%%%%%%%%%%%%%%%%%%%%%%%%%%%%%%%%%
%%%%%%%%%%%%%%%%%%%%%%%%%%%%%%%%%%%%%

\commMV{Make a figure of DM masses vs models}

\commMV{An itemised list of WIMPs, Axions, ultralight, SUSY, ...}

\begin{figure}
    \centering
    \includegraphics{img/chapter_1/DM_model_landscape.pdf}
    \caption{Illustrative landscape of dark matter models}
    \label{fig:DM_models_landscape}
\end{figure}

%%%%%%%%%%%%%%%%%%%%%%%%%%%%%%%%%%%%%
%%%%%%%%%%%%%%%%%%%%%%%%%%%%%%%%%%%%%
\subsection{Dark Matter in an Effective Fields Theory Framework}
%%%%%%%%%%%%%%%%%%%%%%%%%%%%%%%%%%%%%
%%%%%%%%%%%%%%%%%%%%%%%%%%%%%%%%%%%%%


%%%%%%%%%%%%%%%%%%%%%%%%%%%%%%%%%%%%%
%%%%%%%%%%%%%%%%%%%%%%%%%%%%%%%%%%%%%
%%%%%%%%%%%%%%%%%%%%%%%%%%%%%%%%%%%%%
\section{Current Status of Dark Matter Constraints}
%%%%%%%%%%%%%%%%%%%%%%%%%%%%%%%%%%%%%
%%%%%%%%%%%%%%%%%%%%%%%%%%%%%%%%%%%%%
%%%%%%%%%%%%%%%%%%%%%%%%%%%%%%%%%%%%%

%%%%%%%%%%%%%%%%%%%%%%%%%%%%%%%%%%%%%
%%%%%%%%%%%%%%%%%%%%%%%%%%%%%%%%%%%%%
\subsection{Collider Bounds}
%%%%%%%%%%%%%%%%%%%%%%%%%%%%%%%%%%%%%
%%%%%%%%%%%%%%%%%%%%%%%%%%%%%%%%%%%%%

%%%%%%%%%%%%%%%%%%%%%%%%%%%%%%%%%%%%%
%%%%%%%%%%%%%%%%%%%%%%%%%%%%%%%%%%%%%
\subsection{Direct Detection Searches}
%%%%%%%%%%%%%%%%%%%%%%%%%%%%%%%%%%%%%
%%%%%%%%%%%%%%%%%%%%%%%%%%%%%%%%%%%%%

%%%%%%%%%%%%%%%%%%%%%%%%%%%%%%%%%%%%%
%%%%%%%%%%%%%%%%%%%%%%%%%%%%%%%%%%%%%
\subsection{Indirect Detection}
%%%%%%%%%%%%%%%%%%%%%%%%%%%%%%%%%%%%%
%%%%%%%%%%%%%%%%%%%%%%%%%%%%%%%%%%%%%


It is this route that we will follow to explore dark matter EFTs.

%%%%%%%%%%%%%%%%%%%%%%%%%%%%%%%%%%%%%
%%%%%%%%%%%%%%%%%%%%%%%%%%%%%%%%%%%%%
%%%%%%%%%%%%%%%%%%%%%%%%%%%%%%%%%%%%%
\section{Compact Objects as Dark Matter Probes}
%%%%%%%%%%%%%%%%%%%%%%%%%%%%%%%%%%%%%
%%%%%%%%%%%%%%%%%%%%%%%%%%%%%%%%%%%%%
%%%%%%%%%%%%%%%%%%%%%%%%%%%%%%%%%%%%%


Compact objects, namely Neutron Stars and White Dwarfs, offer a unique
laboratory for studying dark matter interactions. Their extreme environments
offer many benefits in comparison to direct detection experiments. 
These include:

\begin{itemize}
\item \textbf{Gravitational focusing of the DM flux.} In the NS case, the 
infalling DM will be boosted to semi-relativistic velocities ($\sim 0.2 - 0.7 c$
depending on the NS mass).
\item 
\end{itemize}


\begin{figure}
    \centering
    \includegraphics[width=0.45\textwidth]{img/chapter_1/ann_heat_NS.pdf}
    \includegraphics[width=0.45\textwidth]{img/chapter_1/kin_heat_NS.pdf}
    \caption{Illustration of DM-induced heating of compact objects. \textbf{Left:} kinetic heating due to DM scattering, raising the temperature to $\sim 1700 \K$. \textbf{Right:} further annihilation heating adding an additional $\sim 800\K$.}
    \label{fig:cartoon_NS_heat}
\end{figure}




