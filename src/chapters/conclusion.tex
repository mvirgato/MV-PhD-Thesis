% \graphicspath{{img/chapter_6/}}

%%%%%%%%%%%%%%%%%%%%%%%%%%%%%%%%%
%%%%%%%%%%%%%%%%%%%%%%%%%%%%%%%%%
%%%%%%%%%%%%%%%%%%%%%%%%%%%%%%%%%
\chapter{Conclusion}
\label{chapter:conclusion}
%%%%%%%%%%%%%%%%%%%%%%%%%%%%%%%%%
%%%%%%%%%%%%%%%%%%%%%%%%%%%%%%%%%
%%%%%%%%%%%%%%%%%%%%%%%%%%%%%%%%%

% \begin{synopsis}
  
% \end{synopsis}
Stars have long been used as laboratories to study dark matter interactions in unique environments. Searching for annihilation signals of dark matter that has been captured within our Sun has proved to be a powerful probe of dark matter interactions, complimentary to direct detection searches.
The natural evolution of these studies are to consider dark matter being captured within compact objects, as they provide many benefits over solar capture. These benefits do not come without a cost, and properly accounting for the unique physics relevant to dark matter scattering within these stellar remnants is a challenging task. However, if we are to make accurate predictions on how sensitive these stars may be to dark matter interactions, then a consistent formalism for calculating the capture and interaction rates is required. The search for this new formalism was the foudnation upon which this thesis was constructed.


In Chapter~\ref{chapter:capture_intro}, we presented our formalism for dark matter capture in compact objects,  building upon the formalism established by Gould for capture in the Sun. The exact expressions for the capture and interaction rates presented here self-consistently incorporate many of the physical processes that, in the past, had either been neglected or were crudely approximated. These include the use of relativistic kinematics to describe both the dark matter and the scattering target; gravitational focusing of the incoming flux of dark matter particles; Pauli blocking of the final state target due to scattering in a degenerate medium; the internal structure of the star by solving the TOV equations coupled to a realistic EoS; suppression of the capture rate for large dark matter masses due to the effects of multiple scattering. 

% In addition, we derived semi-analytic expressions for the interaction rates for matrix elements that are polynomials in the Mandalstam variables $s$ and $t$. These, along with approximations for the capture rates in the non-Pauli blocked regimes, allow for significantly faster computation times of the capture rates compared to numerically evaluating the general expressions. 

We then applied this formalism to the case of dark matter scattering off the leptonic species within compact objects in Chapter~\ref{chapter:capture_leptons}. In neutron stars, these were the electrons and muons that appear in small abundances as a consequence of $\beta$-equilibrium. Despite being significantly less abundant than neutrons, we found that the threshold scattering cross-section that may be probed through future neutron star observations is orders of magnitude below current and upcoming direct detection sensitivities on the dark matter-lepton scattering cross-section. In particular, we find that muons provide better sensitivity to low mass ($\mchi\lesssim 0.1\GeV$) dark matter compared to electron scattering. This is the result of a combination of effects, primarily being that muons are far less degenerate than electrons and hence suffer less Pauli blocking. Additionally, for scalar and pseudoscalar interactions, the electron cross-sections are suppressed relative to the muons by factors of order  $(m_e/m_\mu)^2$. 

In addition, we considered dark matter scattering off the degenerate and relativistic electrons found within white dwarfs.  Here, we made use of the existing observations of white dwarfs in the globular cluster Messier-4 that is predicted to contain a dark matter density of $\rho_\chi = 798\GeV \cm^{-3}$, assuming it was formed in a dark matter subhalo. By comparing the observed luminosity of the oldest (and hence coldest) WD in M4 to the predicted luminosity due to dark matter-induced heating, we were able to constrain the dark matter-electron scattering cross-section, below the current direct detection constraints. 

Chapter~\ref{chapter:capture_baryons} sees us return to dark matter scattering from baryons in neutron stars. At the extreme densities found within the cores of neutron stars, the baryons cannot be treated as a free Fermi gas, and the effects of the strong inter-nucleon interactions must be taken into account. This was done by adopting the QMC EoS that self-consistently incorporates these interactions at the level of the Lagrangian. This leads to all baryons acquiring an effective mass that is less than the vacuum value, and to protons treated as being degenerate throughout even the lightest neutron stars. Additionally, the high momentum transfers achievable due to the dark matter reaching quasi-relativistic energies leads to the structure of the baryons being probed. To account for their finite size, the momentum dependence of the hadronic form factors was reintroduced, suppressing the interaction rates at high momentum transfer. Finally, we saw that the total capture rate in heavy neutron stars can be enhanced when considering scattering from the hyperons that appear in the stellar core. 

In general, we found that properly incorporating the effects of baryon strong interactions and the momentum dependence of their form factors leads to a suppression in the threshold cross-section over a wide range of dark matter masses. In particular, for dark matter heavier in the mass range $\sim 1\GeV-10^6\GeV$, the threshold cross-section is $\sim 10$ times larger than the standard value of $10^{-45}\cm^2$. 

Chapter~\ref{chapter:thermalisation} applies the formalism set up in the preceeding chapters to consider the extent to which dark matter can heat a neutron star. In this chapter, we studied in detail the timescales involved in the thermalisation and heating of an old, cold neutron star due to the capture of dark matter. The timescales are the kinetic heating time, defined as the time required for the dark matter to deposit $99\%$ of its kinetic energy; the thermalisation time; and the time required for capture and annihilation to reach a state of equilibrium. 

Kinetic heating was found to be achievable within a matter of days for all cases we considered. Thermalisation can be achieved on timescales much less than the age of the star for interactions with differential cross-sections independent on the momentum transfer. However, for interactions that depend on $t$, such as operators D2 and D4 (scalar-pseudoscalar and pseudoscalar-pseudoscalar interactions respectively), this time can exceed even the age of the universe for large portions of the dark matter mass range considered. Naively, this would preclude these interactions from being able to further heat the star through their annihilations. 
However, we find that the population of dark matter that has not fully thermalised within the stellar core can begin annihilating at a temperature somewhat higher than the equilibrium one. This allows capture-annihilation equilibrium to be achieved in times much shorter than the age of the star, for all interactions we have considered. 

The results presented in this thesis have shown that compact objects have the potential to be powerful probes of dark matter interactions, complimentary to direct detection experiments. Though we have gone to great lengths to incorporate many of the important pieces of physics relevant for scattering within these stars, there remains, as always, more that can be done. For neutron stars in particular, the importance of scattering with collective modes rather than individual particles has been raised. Perhaps the most important example of this is in treating the neutrons as a superfluid in the inner core of the star, as is expected in many neutron star models. When the momentum transfer is small enough, superfluid phonons are excited are hence the relevant degrees of freedom that need to be considered in the scattering process. Whether these effects can be consistently included within the framework presented in this thesis remains to be seen. 

In the meantime, we optimistically await the day when JWST or future infrared telescopes observe an old, cold, and isolated neutron star, providing us with one more clue towards solving the mystery that is dark matter.