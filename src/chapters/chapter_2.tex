\graphicspath{{img/chapter_2/}}



\chapter{A Primer on Compact Objects}
\label{chapter:compactobjects}

% \begin{synopsis}
% Introduce COs, formation, structure etc...  
% \end{synopsis}
%%%%%%%%%%%%%%%%%%%%%%%%%%%%%%%%%%%%%
%%%%%%%%%%%%%%%%%%%%%%%%%%%%%%%%%%%%%
%%%%%%%%%%%%%%%%%%%%%%%%%%%%%%%%%%%%%

Within the cores of stars exists a delicate balance between the gravitational force of its mass wanting to collapse in on itself, and the outward pressure generated by thermonuclear fusion of light elements. This fusion process begins with the burning of hydrogen to form helium. Eventually, the hydrogen is depleted, allowing gravity to temporarily overcome the outward pressure leading to the core contracting. As this occurs, the gravitational potential energy is converted to thermal energy and the core eventually becomes hot enough to facilitate helium burning. 

This cycle continues as heavier and heavier elements are formed within the ever-increasingly hot stellar core.
If the star is heavy enough, iron will eventually be formed from the burning of silicon. As the fusion of iron nuclei is an endothermic process, it will not occur spontaneously. Whatever the mass of the star, eventually it will no longer be able to support the fusion of these heavier elements. Without a sufficient fuel source, the core will collapse under its own gravity leading to the death of the star.  

What comes after this depends on the mass of the progenitor stars. Very light stars, $\lesssim 0.5 \Msun$, have lifetimes much longer than the age of the universe, and so are uninteresting to our current discussion. Moderately heavy stars, $1\Msun\lesssim M_\star \lesssim 8\Msun$, will continue burning fuel until the outer layers of the star are dispersed as it expands, leaving a carbon-oxygen (CO) core. The core will begin to collapse until the Fermi degeneracy of the ultrarelativistic electrons is great enough to reestablish equilibrium, resulting in a White Dwarf (WD).

Heavy stars, $\gtrsim 8\Msun$, spectacularly end their lives in a type-II supernova event. This occurs when the core of the star exceeds the Chandrashekhar mass of $1.4\Msun$, which cannot be supported by electron degeneracy pressure. The core itself will then collapse, leading to a shockwave that ejects the majority of the mass of the star. All that will remain is an extremely dense core supported by neutron degeneracy pressure, a Neutron Star (NS). If the star was so massive that the gravitational forces overcome even the neutron degeneracy pressure, then the core collapses into a black hole. 

These three stellar corpses (white dwarfs, neutron stars, and black holes) are collectively known as compact objects, as they have masses similar to or larger than our Sun, compressed into much smaller bodies with significantly larger surface gravities. These objects do not have a source of fuel, and spend the rest of their lives cooling down. For the remainder of this thesis, we will only be interested in white dwarfs and neutron stars and refer to these collectively as compact objects, excluding black holes from this term. 

This chapter is dedicated to discussing the structure and composition of these objects, to the level of detail needed from a particle physics persepctive\footnote{As this work is written by a particle physicist, I wish to apologise to my astrophysics colleagues for what is to come.}.


%%%%%%%%%%%%%%%%%%%%%%%%%%%%%%%%%%%%%
%%%%%%%%%%%%%%%%%%%%%%%%%%%%%%%%%%%%%
%%%%%%%%%%%%%%%%%%%%%%%%%%%%%%%%%%%%%
\section{Internal structure}
%%%%%%%%%%%%%%%%%%%%%%%%%%%%%%%%%%%%%
%%%%%%%%%%%%%%%%%%%%%%%%%%%%%%%%%%%%%

In this thesis, we will only be considering spherically symmetric and static stars. 
The internal structure of 

For non-relativistic stars, such as our Sun, the structure equations are
rather simple, them being
\begin{align}
\frac{d M}{dr}  &= 4 \pi r^2,\\
\frac{d P }{dr} &= -\rho(r) G M(r) r^2,
\end{align}
where $M$ is the mass of the star at radius $r$, $P$ is the pressure,
and $\rho(r)$ is the density.
The first equation is known as the mass equation, and the second is the
condition for hydrostatic equilibrium. 

We are instead interested in compact objects, where the extreme densities 
of these stars 

In order to solve this system, 
The equation of state describes the relation between the pressure 
and energy of the constituent matter.


%%%%%%%%%%%%%%%%%%%%%%%%%%%%%%%%%%%%%
%%%%%%%%%%%%%%%%%%%%%%%%%%%%%%%%%%%%%
\subsection{White Dwarfs}
%%%%%%%%%%%%%%%%%%%%%%%%%%%%%%%%%%%%%
%%%%%%%%%%%%%%%%%%%%%%%%%%%%%%%%%%%%%

FMT equation of state

%%%%%%%%%%%%%%%%%%%%%%%%%%%%%%%%%%%%%
%%%%%%%%%%%%%%%%%%%%%%%%%%%%%%%%%%%%%
\subsection{Neutron Stars}
%%%%%%%%%%%%%%%%%%%%%%%%%%%%%%%%%%%%%
%%%%%%%%%%%%%%%%%%%%%%%%%%%%%%%%%%%%%


Beta Equlibrium
%%%%%%%%%%%%%%%%%%%%%%%%%%%%%%%%%%%%%
%%%%%%%%%%%%%%%%%%%%%%%%%%%%%%%%%%%%%
%%%%%%%%%%%%%%%%%%%%%%%%%%%%%%%%%%%%%
\section{Observational Status}
%%%%%%%%%%%%%%%%%%%%%%%%%%%%%%%%%%%%%
%%%%%%%%%%%%%%%%%%%%%%%%%%%%%%%%%%%%%
%%%%%%%%%%%%%%%%%%%%%%%%%%%%%%%%%%%%%

%%%%%%%%%%%%%%%%%%%%%%%%%%%%%%%%%%%%%
%%%%%%%%%%%%%%%%%%%%%%%%%%%%%%%%%%%%%
\subsection{White Dwarfs}
%%%%%%%%%%%%%%%%%%%%%%%%%%%%%%%%%%%%%
%%%%%%%%%%%%%%%%%%%%%%%%%%%%%%%%%%%%%

%%%%%%%%%%%%%%%%%%%%%%%%%%%%%%%%%%%%%
%%%%%%%%%%%%%%%%%%%%%%%%%%%%%%%%%%%%%
\subsection{Neutron Stars}
%%%%%%%%%%%%%%%%%%%%%%%%%%%%%%%%%%%%%
%%%%%%%%%%%%%%%%%%%%%%%%%%%%%%%%%%%%%

