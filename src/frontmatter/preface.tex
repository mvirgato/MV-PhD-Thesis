\begin{preface}
This thesis is based on a series of consecutive papers that each builds upon the last, culminating in Ref.~\cite{Bell:2023ysh_dec_ThermalizationAnnihilationDark}. As a result, Refs.~\cite{Bell:2020jou_sep_ImprovedTreatmentDark, Bell:2020lmm_mar_ImprovedTreatmentDark, Bell:2020obw_sep_NucleonStructureStrong, Anzuini:2021lnv_nov_Improvedtreatmentdark} have a considerable amount of overlap in their introductory sections in order to ensure each paper was a self-contained piece of literature. As such, simply putting each paper into its own chapter of this thesis would not result in a cohesive work. 
These papers were all written in collaboration with my supervisor, Prof. Nicole Bell, as well as Dr. Girogio Busoni, and Dr. Sandra Robles. The original motivation for beginning this series of papers should be credited to them.

It is therefore simpler to discuss the authors contribution to each paper before discussing the content of each chapter.

\begin{itemize}
    \item In Ref.~\cite{Bell:2020jou_sep_ImprovedTreatmentDark}, which introduces the capture formalism for neutron stars, the authors main contributions were in writing sections of the text and working on the numerical code originally written by Giorgio Busoni. 
    \item Ref.~\cite{Bell:2020lmm_mar_ImprovedTreatmentDark}
    \item Ref.~\cite{Bell:2020obw_sep_NucleonStructureStrong} the author worked on the numerical calculations, with little input in the text. As such. As such, very little of this work is included in this thesis.
    \item Ref.~\cite{Anzuini:2021lnv_nov_Improvedtreatmentdark} several numerical calculations and wrote several sections discussing the results. 
    \item All numerical calculations of Ref.~\cite{Bell:2023ysh_dec_ThermalizationAnnihilationDark} were performed by the author. In addition, the first draft of all sections save section 2 were originally written by the author.
    \item Ref.~\cite{Bell:2021fye_oct_Improvedtreatmentdark} can be divided into two parts that discuss capture from ions and electrons in white dwarfs. The calculations and discussions involving DM-ion interactions are the work of Maura Quezada-Ramirez, and are not included in this work. The remaining sections on DM electron interactions are the work of the author. In addition, the discussion and calculation of the white dwarf structure and equation of state were performed by the author.
\end{itemize}
In addition, the plots presented in the papers were all made by Sandra Robles. Only when necessary were these figures remade before inclusion into this thesis.
Any part of the papers not written by the author have been rewritten before being included in this thesis. 

The chapters of this thesis are as follows:
\begin{itemize}
    \item Chapter~\ref{chapter:introduction} is an original review of the literature.
    \item Chapter~\ref{chapter:compactobjects} is an original introduction to compact objects. We detail their internal structure in general before having dedicated discussions on white dwarfs and neutron stars. The discussion on the white dwarf equation of state and observational status is in large part taken from early sections of Ref.~\cite{Bell:2021fye_oct_Improvedtreatmentdark} .
    \item Chapter~\ref{chapter:capture_intro} begins with an original overview of the Gould formalism for DM capture in the Sun. The remainder of the chapter discusses the capture of dark matter in compact objects and is based largely on Ref.~\cite{Bell:2020jou_sep_ImprovedTreatmentDark}, with some results of Refs.~\cite{Bell:2020lmm_mar_ImprovedTreatmentDark, Anzuini:2021lnv_nov_Improvedtreatmentdark}. 
\end{itemize}
\end{preface}
