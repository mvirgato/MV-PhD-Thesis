\begin{abstract}%
Compact objects, namely neutron stars and white dwarfs, are expected to play an important role in the next generation of astrophysical probes into the nature of dark matter. 
% They are highly efficient in capturing the ambient dark matter in the galactic halo and can provide a myriad of signals that may be searched for with current and upcoming telescopes. 
One such signal that has been of recent interest is the heating of these stars due to the capture, thermalisation, and annihilation of dark matter.
% is the heating of these stars due to dark matter being captured within them. 
In order to accurately predict how sensitive these stars can be to dark matter interactions, we require a self-consistent description of the capture and interaction rates that account for the physical processes that arise due to the extreme nature of these objects. 

Chapter~\ref{chapter:introduction} serves as a review of dark matter. We discuss the observational evidence in support of its existence and the current experimental constraints on its properties. Special attention is paid to the limits that arise from considering dark matter captured within the Sun.

Chapter~\ref{chapter:compactobjects} then provides an overview of compact objects at the level required for this thesis. We discuss the internal structure of these stars and the details of the relevant equations of state that describe their constituent matter.

The formalism for dark matter capture in compact objects is introduced in Chapter~\ref{chapter:capture_intro}, which builds upon Gould's original formalism for capture in the Sun. The formalism presented correctly incorporates the effects of gravitational focusing, a fully relativistic treatment of the scattering, Pauli blocking of the final state targets, and multi-scattering effects. We highlight these effects in the context of dark matter scattering off neutron targets within a neutron star.

We then apply this formalism to the case of dark matter scattering with leptonic species in compact objects in Chapter~\ref{chapter:capture_leptons}. In neutron stars, these are the subdominant electron and muon species. We estimate the sensitivity future observations may have to the dark matter-lepton scattering cross-section. Interestingly, we find muons can provide the strongest sensitivities to low-mass dark matter despite being less abundant than the electrons. 
In white dwarfs, the electrons are extremely relativistic and degenerate, hence requiring this new formalism to model the capture and interaction rates correctly. We use observations of white dwarfs in the globular cluster Messier 4, which we assume to be located in a DM sub-halo, to constrain the dark matter-electron scattering cross-section. In addition, we discuss how accounting for the finite temperature of the stars leads to several interesting effects. These include a boost to the capture rate for sub-GeV dark matter, as well as allowing the possibility that the dark matter will evaporate from the star if it has a light enough mass.

Chapter~\ref{chapter:capture_baryons} then considers dark matter scattering on hadronic targets in neutron stars, including neutrons, protons and hyperons. While the leptonic species can safely be treated as a free Fermi gas, this is not the case for the baryonic species, as they undergo strong interactions amongst themselves. These effects are accounted for via the adoption of the QMC equation of state for dense nuclear matter. These strong interactions induce effective masses for the nucleons, which are less than their value in vacuum. In addition, the momentum transfers can be large enough that the finite size of the hadron needs to be accounted for by including the momentum dependence of the hadronic form factors. 

Chapter~\ref{chapter:thermalisation} applies the results of the previous chapters to discuss dark matter induced heating of neutron stars. In order to determine the extent to which dark matter can heat the star, we pay particular attention to the timescales involved in each stage of the heating process. This occurs in two stages, with the dark matter first depositing its kinetic energy as it continues to scatter after being captured, and later its mass-energy as it annihilates within the core of the star. In order for this annihilation heating to be appreciable, the dark matter must reach a state of capture-annihilation equilibrium within the lifetime of the star. Thermalisation has typically been assumed to be required for this state to be achieved; however, we find that this is not the case in neutron stars. We generalise the capture-annihilation criteria to account for the case of partially thermalised dark matter. Importantly, we find that this allows interactions that would not fully thermalise within the age of the Universe to reach this steady state and maximally heat the star.

Finally, we summarise the key findings of this thesis in Chapter~\ref{chapter:conclusion}, where we provide our concluding remarks and notes on the future directions for this work.

  
\end{abstract}
