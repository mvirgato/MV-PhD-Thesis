\begin{abstract}%
  Compact objects, namely neutron stars and white dwarfs, are expected to play an important role in the next generation of astrophysical probes into the nature of dark matter. They are highly efficient in capturing the ambient dark matter in the galactic halo, and can provide a myriad of signals that may be searched for with current and upcoming telescopes. To provide accurate predictions of how sensitive these observations will be to the interactions, a description of the capture and interaction rates that account for the extreme natures of stars is required. 

  Chapter~\ref{chapter:introduction} serves as a review of the current status of dark matter. We discuss the observational evidence in support of its existence and the current experimental constraints on its properties. Special attention is paid to the limits set due to dark matter capture within the Sun.

  Chapter~\ref{chapter:compactobjects} then provides an overview of the necessary information about compact objects that is required in this thesis. We discuss the internal structure of these stars, and the details of the relevant equations of state that describe their constituent matter.

  The formalism for dark matter capture in compact objects in introduced in Chapter~\ref{chapter:capture_intro}, where it is built upon Gould's original formalism for capture in the Sun.
\end{abstract}
